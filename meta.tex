\documentclass{article}
\usepackage[utf8]{inputenc}
\usepackage{amsmath}

\title{Meta Analysis}
\author{Travis Andersen }
\date{November 2020}

\begin{document}

\maketitle

\section{Frameworks}

\begin{itemize}

    \item Fixed effect: The studies are assumed to investigate the same population, use the same variable and outcome definitions, etc.

    \item Random effects: The studies are assumed to be heterogeneous.  

\end{itemize}

\section{Models}

We are considering the case where there are k independent studies done, resulting in k 2 x 2 tables as shown below. The parameter of interest in these cases is often the odds ratio, defined as $OR = \frac{a/b}{c/d} = \frac{ad}{bc}$. We are comparing different methods of pooling the odds ratios.
\begin{table}[h!]
\centering
    \begin{tabular}{|c|c|}
        \hline
        a & b \\
        \hline
        c & d \\
        \hline
    \end{tabular}
\caption{2 x 2 table}
\end{table}

\begin{itemize}
    \item Inverse variance: Assumes fixed effect framework.
        \begin{equation}
            \hat{OR}_{IV} = \frac{\sum_{i=1}^k \text{Var}(\frac{a_i d_i}{b_i c_i})^{-1} \frac{a_i d_i}{b_i c_i}}{\sum_{i=1}^k \text{Var}(\frac{a_i d_i}{b_i c_i})^{-1}} 
        \end{equation} 

    \item DerSimonian-Laird: Adaption of inverse variance for random effects framework.
        \begin{equation}
            \hat{OR}_{DL} =
        \end{equation} 
        
    \item Mantel-Haenszel: Assumes fixed effect framework. Better at binary outcomes than inverse variance.
        \begin{equation}
            \hat{OR}_{MH} = \frac{\sum_{i=1}^k \frac{a_i d_i}{n_i}}{\sum_{i=1}^k \frac{b_i c_i}{n_i}}, \text{where } n_i = a_i + b_i + c_i + d_i
        \end{equation}    
        
    \item Peto: Can cause bias when samples are very unbalanced. Different from odds ratio. 
        \begin{equation}
            \hat{\psi}_{pooled} = \text{exp}(\frac{\sum_{i=1}^k (O_i - E_i)}{\sum_{i=1}^k V_i})
        \end{equation} 
        O, E, and V are defined as follows (where $n = a+b+c+d$):
        \begin{equation}
            O = a
        \end{equation}
        \begin{equation}
            E = \frac{(a+b)(a+c)}{n}
        \end{equation}
        \begin{equation}
            V = \frac{(a+b)(c+d)(a+c)(b+d)}{n^2(n-1)}
        \end{equation}
        
\end{itemize}

\end{document}
